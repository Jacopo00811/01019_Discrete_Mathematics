\section{Exercise D}
In order to show that the equalities hold we used propositional logic formulas and the truth table method.

\begin{enumerate}[label=\alph*)]

\item $ A \cup (B \cap C) = (A \cup B) \cap (A \cup C) $ \\
Translated into propositional logic becomes: \\ 
$ A \lor (B \land C) = (A \lor B) \land (A \lor C) $ \\
To which we can apply the distributive property of $\lor$ over $\land$ on the left hand side: \\
$ (A \lor B) \land (A \lor C) = (A \lor B) \land (A \lor C) $

\item $ A-(A \cap B) = A-B $ \\
Translated into propositional logic becomes: \\
$ A \land \neg(A \land B) = A \land \neg B $ \\
To which we can apply De Morgan's law on the left hand side: \\
$ A \land (\neg A \lor \neg B) = A \land \neg B $ \\
And then apply the distributive property of $\land$ over $\lor$ on the left hand side: \\
$ (A \land \neg A) \lor (A \land \neg B) = A \land \neg B $ \\
From here we can see that the block $(A \land \neg A)$ is always false, thus the following $\lor$ condition only depends on $A \land \neg B$. We can rewrite the equation in this form: \\
$ A \land \neg B = A \land \neg B $ 

\item $ A \cap (A \cup B) = A $  \\
Translated into propositional logic becomes: \\
$ A \land (A \lor B) = A $ \\
From here we can already say that the formula $A \land (A \lor B)$ is true only when $A$ is true. As a proof we construct the table from which we can see that $A \land (A \lor B)$ gets a true value only when $A$ is set to true.

\begin{center}
\begin{tabular}{ |c|c||c|c| } 
\hline
A & B & $A \lor B$ & $A \land (A \lor B)$  \\
\hline
\hline
T & T & T & T\\
\hline
T & F & T & T\\
\hline
F & T & T & F\\
\hline
F & F & F & F\\
\hline
\end{tabular}
\end{center}

 So the formula $A \land (A \lor B) = A$ is only dependent on A and can be rewritten as: \\
 $A=A$.
 
\end{enumerate}